\documentclass[11pt]{article}
\usepackage[margin=1in]{geometry}
\usepackage{graphicx}
\usepackage{booktabs}
\usepackage{amsmath}
\usepackage{hyperref}

\title{Chicago PPLTT and ChatGPT Subscriptions}
\author{}
\date{\today}

\begin{document}
\maketitle

\section{Background}

Chicago's Personal Property Lease Transaction Tax (PPLTT) applies to AI subscription services including ChatGPT Plus. The tax rate increased from 9\% (October 2023) to 11\% (January 2025).

\section{Pass-Through Evidence}

Figure~\ref{fig:price} shows the median transaction amount for ChatGPT subscriptions in Chicago (zip3 606). The base ChatGPT Plus price is \$20/month. Full pass-through would imply prices of \$21.80 at 9\% and \$22.20 at 11\%.

\begin{figure}[htbp]
    \centering
    \includegraphics[width=0.9\textwidth]{chicago_chatgpt_median_price.png}
    \caption{Median transaction amount for ChatGPT subscriptions in Chicago. Horizontal lines indicate expected prices under full pass-through. The IQR band shows the 25th--75th percentile range.}
    \label{fig:price}
\end{figure}

Observed median prices are slightly below full pass-through (\$21.36 vs \$21.80 during the 9\% period; \$21.95 vs \$22.20 during the 11\% period), suggesting pass-through is high but not complete.

\section{Quantity Effects}

\subsection{Raw Time Series}

Figure~\ref{fig:raw} compares log transactions in Chicago to the simple mean of 93 size-matched control zip3s. Controls are selected as zip3s with transaction counts within 50\% of Chicago's count during January--June 2023.

\begin{figure}[htbp]
    \centering
    \includegraphics[width=0.9\textwidth]{chicago_raw_counts.png}
    \caption{Raw time series of log transactions: Chicago vs.\ mean of 93 size-matched control zip3s.}
    \label{fig:raw}
\end{figure}

\subsection{Event Study}

Figure~\ref{fig:did} presents event study estimates from a difference-in-differences specification:
\[
\log(\text{transactions}_{zt}) = \sum_{\tau \neq \text{Sep 2023}} \beta_\tau \cdot \mathbf{1}[\text{Chicago}] \cdot \mathbf{1}[\text{month} = \tau] + \alpha_z + \gamma_t + \varepsilon_{zt}
\]
Standard errors are clustered at the zip3 level.

\begin{figure}[htbp]
    \centering
    \includegraphics[width=0.9\textwidth]{chicago_did.png}
    \caption{Event study coefficients for Chicago relative to 93 size-matched control zip3s. Reference period is September 2023. Error bars show 95\% confidence intervals. Light red shading indicates the 9\% tax period; darker red indicates 11\%.}
    \label{fig:did}
\end{figure}

The pooled DiD coefficient is $-0.19$ (se $= 0.02$), implying Chicago subscriptions fell approximately 17\% relative to controls after the tax. Separating by tax rate:
\begin{itemize}
    \item 9\% period (Oct 2023--Dec 2024): coefficient $\approx -0.08$, elasticity $\approx -1$
    \item 11\% period (Jan 2025+): coefficient $\approx -0.33$, elasticity $\approx -3$
\end{itemize}

Pre-trends are not perfectly flat (joint F-test rejects), but magnitudes are small (mean $|\text{coef}| = 0.05$) relative to post-treatment effects.

\section{Discussion}

The evidence suggests:
\begin{enumerate}
    \item Pass-through is high but slightly incomplete
    \item Quantity declined relative to controls, with the effect growing at higher tax rates
    \item Implied elasticity is around $-1$ at 9\% and $-3$ at 11\%, consistent with demand becoming more elastic at higher prices
\end{enumerate}

\end{document}
