\documentclass[11pt]{article}
\usepackage[margin=1in]{geometry}
\usepackage{booktabs}
\usepackage{graphicx}
\usepackage{amsmath}

\title{Synthetic Control Analysis: Chicago PPLTT on ChatGPT Subscriptions}
\date{January 2026}

\begin{document}
\maketitle

\section{Method}

We estimate the effect of Chicago's 9\% Personal Property Lease Transaction Tax (PPLTT) on ChatGPT subscriptions using synthetic control. The method constructs a weighted average of untreated ZIP3s to match Chicago's pre-treatment trajectory, then compares actual vs.\ synthetic outcomes post-treatment.

\textbf{Matching:} Weights fitted on pre-treatment outcomes only (March--September 2023, 7 months). The optimizer minimizes squared prediction error subject to convex hull constraints:
\begin{align}
\min_{w} \quad & \sum_{t < T_0} \left( Y_{606,t} - \sum_{j} w_j Y_{j,t} \right)^2 \\
\text{s.t.} \quad & \sum_j w_j = 1, \quad w_j \in [0,1] \; \forall j
\end{align}

\textbf{Outcome:} $\log(\text{unique cardids with \$15--\$25 transaction})$ per ZIP3-month.

\section{Sample}

\begin{itemize}
    \item \textbf{Treated unit:} ZIP3 606 (Chicago)
    \item \textbf{Treatment date:} October 2023
    \item \textbf{Donor pool:} 19 ZIP3s with unique users within 10\% of Chicago (181 users) in March--June 2023
    \item \textbf{Panel:} Constant individual panel (cardlinkids active in all 70-day windows)
    \item \textbf{Transaction filter:} \$15--\$25 (ChatGPT Plus subscription range)
    \item \textbf{Sample period:} March 2023 -- November 2024
\end{itemize}

\section{Results}

\begin{table}[h]
\centering
\begin{tabular}{lc}
\toprule
Metric & Value \\
\midrule
Pre-period RMSE & 0.018 \\
Pre-tax gap (Chicago $-$ Synth) & $+0.001$ \\
Post-tax gap (Chicago $-$ Synth) & $-0.112$ \\
\textbf{Treatment effect} & \textbf{$-$11.2 log points} \\
\bottomrule
\end{tabular}
\end{table}

The pre-treatment fit is near-perfect. Post-tax, Chicago falls below the synthetic control by approximately 11\%, and this gap persists through 2024.

\begin{figure}[h]
\centering
\includegraphics[width=0.9\textwidth]{chicago_synth_control.png}
\caption{Chicago vs.\ synthetic control. Weights fitted on pre-treatment outcomes (March--September 2023). Shaded region indicates post-tax period.}
\end{figure}

\section{Donor Weights}

Eight ZIP3s receive weight $\geq 1\%$:

\begin{table}[h]
\centering
\begin{tabular}{ccl}
\toprule
ZIP3 & Weight & Location (?) \\
\midrule
281 & 22.5\% & Houston, TX (?) \\
953 & 20.2\% & Stockton, CA (?) \\
280 & 15.8\% & Miami, FL (?) \\
890 & 15.8\% & Gulfport, MS (?) \\
957 & 13.4\% & Fresno, CA (?) \\
232 & 5.5\% & Cincinnati, OH (?) \\
208 & 3.5\% & Washington, DC (?) \\
775 & 3.2\% & Lubbock, TX (?) \\
\bottomrule
\end{tabular}
\caption{ZIP3-to-city mappings need verification against USPS data.}
\end{table}

\section{Interpretation}

The 9\% Chicago PPLTT reduced the number of unique ChatGPT Plus subscribers by approximately 11\% relative to the synthetic counterfactual. This is the extensive margin effect---fewer distinct cardholders making subscription-priced transactions.

At a \$20 subscription price, a 9\% tax is \$1.80/month. An 11\% reduction in subscribers implies a demand elasticity on the order of $-1.2$, though this is rough given the level shift rather than continuous price variation.

\section{Technical Notes}

\begin{itemize}
    \item \textbf{Matching on outcomes only (limitation):} Currently matching on pre-treatment outcomes only. This is non-standard---Abadie et al.\ recommend matching on both outcomes and covariates. Without covariate matching, the synthetic may be fitting noise in the pre-period. Ideal covariates for matching ZIP3 606: college education rate, median income, total population. ZIP3-level demographics not yet available (ask Matt Noto?).
    \item \textbf{Donor pool restriction:} Using all 751 ZIP3s caused optimizer convergence issues (weights concentrated on 3 ZIP3s, poor fit). Restricting to size-matched donors (within 10\% of Chicago) yields stable results.
    \item \textbf{All weights used:} The synthetic is computed using all donor weights (summing to 1), not just the $>1\%$ subset shown above.
\end{itemize}

\end{document}
