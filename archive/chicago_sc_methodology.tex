\documentclass[12pt]{article}
\usepackage[margin=1in]{geometry}
\usepackage{amsmath}
\usepackage{booktabs}
\usepackage{natbib}
\usepackage{hyperref}

\title{Synthetic Control Estimation of Chicago's Lease Tax
on ChatGPT Subscriptions}
\author{}
\date{\today}

\begin{document}
\maketitle

\section{Setting}

Chicago imposes a 9\% Personal Property Lease Transaction Tax
(PPLTT) on digital subscriptions, including ChatGPT Plus.
The tax took effect in October 2023.
We estimate the effect on ChatGPT subscription adoption
using synthetic control methods.

\section{Identification Strategy}

We construct a synthetic Chicago from a weighted average
of untreated ZIP3 regions.
Following \citet{abadie2010synthetic},
we match on pre-treatment covariates only---not
pre-treatment outcomes.
This approach finds regions that \emph{look like} Chicago
demographically, rather than regions that happened to have
similar ChatGPT trends in the pre-period.

Matching on pre-treatment outcomes risks overfitting
to idiosyncratic noise.
If Chicago had an unusually good or bad month for
unrelated reasons, outcome-matching would select
donors that share that noise.
Covariate-only matching avoids this problem.

\section{Covariate Selection}

We match on three ZIP3-level characteristics from the
2022 American Community Survey (5-year estimates).
Each has a clear economic rationale for predicting
ChatGPT adoption.

\subsection{Education (Percent College-Educated)}

ChatGPT's value proposition centers on knowledge work:
writing, coding, analysis, research.
College-educated workers hold jobs where these tasks
are common.
Education also correlates with technological literacy
and willingness to adopt new tools.

\subsection{Income (Percent of Households Earning \$100k+)}

ChatGPT Plus costs \$20 per month---discretionary spending.
The share of high-income households captures
ability to pay for productivity subscriptions.
We use this measure rather than median income because
medians cannot be aggregated from ZCTA to ZIP3;
proportions can.

\subsection{Age (Percent Aged 18--34)}

Younger adults adopt new technologies faster.
This pattern appears across smartphones, social media,
and software tools.
AI assistants follow the same adoption curve.

\subsection{Covariates Not Included}

\paragraph{Broadband access.}
The 2022 ACS shows 85\%+ broadband penetration
in most ZIP3s, including Chicago (85.2\%).
Limited variation remains.
Broadband likely proxies for urban/rural status
rather than directly driving ChatGPT adoption.
ChatGPT also works on mobile networks.

\paragraph{STEM occupation share.}
STEM workers are heavy ChatGPT users
(code generation, technical writing).
However, STEM share correlates strongly with
education and income.
Conditional on those two covariates,
STEM adds little independent information.

\section{Data}

\paragraph{Outcome.}
Log unique cardholders with a \$15--25 ChatGPT transaction
per ZIP3-month.
This range captures Plus subscriptions (\$20 base price)
with variation from taxes and currency conversion.

\paragraph{Treatment.}
ZIP3 606 (Chicago), October 2023 onward.

\paragraph{Donor pool.}
All other ZIP3s with complete data
March 2023 through November 2024.
We require a balanced panel: ZIP3s must have
at least one ChatGPT transaction in all 21 months.
This drops approximately 11\% of ZIP3s,
primarily small or rural areas with zero transactions
in some months.
Since Chicago is a large urban area,
these dropped ZIP3s would be poor donors regardless;
excluding them does not bias the synthetic control.

\paragraph{Demographics.}
American Community Survey 2022 5-year estimates,
aggregated from ZCTA to ZIP3 using population weights.

\section{Estimation}

We estimate synthetic control weights using
Stata's \texttt{synth} command
\citep{abadie2010synthetic, abadie2015comparative}.
The command finds non-negative weights summing to one
that minimize the distance between treated and
synthetic control covariates.
See the Stata documentation for implementation details.

\subsection{Covariate-Only Matching: A Negative Result}

We first attempted pure covariate matching
(demographics only, no pre-treatment outcomes).
The synthetic control achieves good demographic balance:
41\% college vs.\ 43\% synthetic,
35\% high-income vs.\ 38\% synthetic,
9.8\% young vs.\ 10.4\% synthetic.

Despite this balance, the synthetic has far fewer
ChatGPT users than Chicago---a 2.2 log-point gap
(approximately 9$\times$ fewer users) even before treatment.
Demographics alone do not explain cross-sectional
variation in ChatGPT adoption.
Unobserved factors (tech company presence,
university density, early-adopter networks)
drive usage levels.

\subsection{Hybrid Matching}

Given this negative result, we add the pre-treatment
average of log users to the matching variables.
This single scalar captures the level of ChatGPT adoption
without fitting to month-by-month noise in the pre-period.
The hybrid approach matches on:
\begin{itemize}
\item Percent college-educated
\item Percent high-income households
\item Percent aged 18--34
\item Mean log users, March--September 2023
\end{itemize}

\bibliographystyle{aer}
\bibliography{references}

\end{document}
